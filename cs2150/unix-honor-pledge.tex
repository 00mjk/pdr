\documentclass{article}
\usepackage{palatino}
\usepackage[letterpaper,left=0.8in,right=0.8in,top=0.8in,bottom=0.8in]{geometry}

\pagestyle{empty}

\begin{document}

\Large

\noindent Name (print): \line(1,0){250} \hspace{0.25in} UVa userid:
\line(1,0){50}

\vspace{0.5in}

\noindent {\huge Unix Honor Pledge for CS 2150, (semester)}

\vspace{0.5in}

\noindent On my honor as a student at the University of Virginia, I
agree, unless specifically given permission otherwise:

\begin{itemize}

\item To not use any integrated development environment (IDE) for the
  development of C++ or assembly programs for the work for this
  course. These include, but are not limited to, such development
  environments such as Eclipse, Netbeans, Xcode, Geany, Visual Studio,
  Atom, etc.

\begin{itemize}

\item This applies to IDEs on any platform, such as Microsoft Windows,
  Mac OS X, Unix, etc.

\end{itemize}

\item To develop my course programming work on Unix or Unix-like
  system. These systems include Linux (any variant), FreeBSD, Solaris,
  Unix systems installed on a virtual machine running on another OS
  (such as Windows or Mac OS X), remote Unix systems, and others as
  discussed in lecture.

\begin{itemize}

\item IDEs cannot be used in such systems, however, as that defeats
  the purpose.

\item Mac OS X is allowed as long as the Unix-based features are used,
  and IDEs are not used.

\end{itemize}

\item If there is any doubt about the applicability of this pledge, I
  will ask before assuming.

\end{itemize}

\vspace{0.25in}

\noindent Failure to abide by this agreement will mean an immediate
failure for the course, and the raising of honor charges.

\vspace{0.5in}

\noindent Signature: \line(1,0){250}

\end{document}
